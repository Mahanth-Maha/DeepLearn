\documentclass[10pt,twocolumn]{article}
\usepackage[margin=0.70in]{geometry}
\usepackage{lipsum,mwe,abstract}
\usepackage[T1]{fontenc} 
\usepackage[english]{babel} 
\usepackage{biblatex}
% \addbibresource{ref.bib} 
\bibliography{ref}

\usepackage{fancyhdr} % Custom headers and footers
\pagestyle{fancyplain} % Makes all pages in the document conform to the custom headers and footers
\fancyhead{} 
\fancyfoot[C]{\thepage} % Page numbering for right footer
\usepackage{lipsum}
\setlength\parindent{0pt} 

\usepackage{amsmath,amsfonts,amsthm} % Math packages
\usepackage{wrapfig}
\usepackage{graphicx}
\usepackage{float}
\usepackage{subcaption}
\usepackage{comment}
\usepackage{enumitem}
\usepackage{cuted}
\usepackage{sectsty} % Allows customizing section commands
\allsectionsfont{\normalfont \normalsize \scshape} % Section names in small caps and normal fonts

\renewenvironment{abstract} % Change how the abstract look to remove margins
{\small
\begin{center}
  \bfseries \abstractname\vspace{-.5em}\vspace{0pt}
\end{center}
\list{}{%
\setlength{\leftmargin}{0mm}
\setlength{\rightmargin}{\leftmargin}%
}
\item\relax}
{\endlist}

\makeatletter
\renewcommand{\maketitle}{\bgroup\setlength{\parindent}{0pt} % Change how the title looks like
\begin{flushleft}
  \textbf{\@title}
  \@author \\ 
  \@date
\end{flushleft}\egroup
}
\makeatother

%% ------------------------------------------------------------------- 

\title{
  \Large Optimization Techniques Report  \\
  [10pt] 
  }
  \date{\today}
  \author{Mahanth Yalla}
  
  \begin{document}

  \twocolumn[ \maketitle ]
  
  % --------------- ABSTRACT
  \begin{abstract}
    Experimented on different optimization Techniques and findings are documented in this document.
  \end{abstract}
  
  \rule{\linewidth}{0.5pt}
  
  % --------------- MAIN CONTENT
  
  % If you want to put a big figure over two columns, you can use \begin{figure*}
  % The figure will be sent on top of the next page 
  % \begin{figure*}   
  %     \centering
  %     \includegraphics[width=0.9\textwidth]{Dummy_fig.png}
  %     \caption{Caption}
  % \end{figure*}
  % You can also use the following command, but it can be more tricky to place the figure exactly where you want
  % \begin{strip}
  %     \centering\noindent
  %     \includegraphics[width=0.9\textwidth]{Dummy_fig.png}
  %     \captionof{figure}{Caption}
  % \end{strip}
  \section{Gradient Descent}
  Gradient descent is a way to minimize an objective function $J(\theta)$ parameterized by a model's 
  parameters $\theta \in \mathbb{R}^d$ by updating the parameters in the opposite direction of the gradient 
  of the objective function $ \nabla_{\theta}J(\theta)$ with respect to the parameters. The learning rate $\eta$ 
  determines the size of the steps we take to reach a (local) minimum. In other words, we follow the direction of the slope of the surface created by the objective function downhill until we reach a valley.
  \cite{DBLP:journals/corr/Ruder16}
  
  \section{Variations of Gradient Descent}
  \begin{itemize}
    \item Stochastic (SGD)
    \item Mini-batch SGD 
    \item Momentum 
    \item Nesterov Accelerated (NAG)
    \item Adagrad
    \item Adadelta
    \item RMSprop
    \item Adam
    \item AdaMax
    \item Nadam
    \item AMSGrad
    \item AdamW
    \item Yogi
    \item RAdam
    \item Lookahead
  \end{itemize}


  
  
% \bibliographystyle{plain}
\printbibliography

 \end{document}
 



